\documentclass[../style/orpheus,fontsize=11pt]{scrlttr2}

\begin{document}

%\enlargethispage{5cm}

\begin{letter}{
\noindent Dr. Isolde-Dietrich-Stiftung \\
DSZ - Deutsches Stiftungszentrum im Stiftungsverband f\"ur die Deutsche Wissenschaft \\
Barkovenallee 1\\
45239 Essen
}


\setkomavar{subject}{Talentierte Physiksch\"uler aus ganz Deutschland trainieren f\"ur die IPhO -- Unterst\"utzung gesucht}

\opening{Sehr geehrte Damen und Herren,}

seit mehr als 42 Jahren schon findet j\"ahrlich der weltweit gr\"o\ss te Physikwettbewerb f\"ur Sch\"uler der Sekundarstufe II statt, die IPhO.  Die International Physics Olympiad (IPhO) bringt einmal im Jahr mehr als 400 talentierte und motivierte junge Physiksch\"uler aus aller Welt zusammen.  Die IPhO baut internationale soziale Netze, f\"ordert die Eliten von morgen und ehrt die Besten der Besten.

Damit die Deutsche Nationalmannschaft auch weiter zu den Besten weltweit geh\"ort und der enormen Konkurrenz aus dem asiatischen Raum etwas entgegenzusetzen hat, ist es notwendig, die Sch\"uler in einem aufwendigen Auswahlverfahren auszuw\"ahlen und sie gezielt und intensiv auf dieses und die internationalen Anforderungen vorzubereiten.  Zu diesem Zweck unterst\"utzt der Orpheus e.\, V. (Organisationsgruppe Physik f\"ur europ\"aische Sch\"uler und Studenten) das Auswahlverfahren f\"ur das Team der Deutschen Nationalmannschaft in vielf\"altiger Weise.  Insbesondere werden jedes Jahr Vorbereitungsseminare f\"ur die Teilnehmer der zweiten Runde des Auswahlverfahrens durchgef\"uhrt, da diese schwierigste Runde die gr\"o\"ste H\"urde f\"ur die Sch\"uler darstellt.

Der Orpheus Verein besteht aus ehemaligen Teilnehmern des Auswahlverfahrens und des internationalen Wettbewerbs, die die Anforderungen genau kennen und mittlerweile gro\ss e Erfahrung in der F\"orderung von begabten und hochbegabten jungen Sch\"ulern haben.  Die Sch\"uler sind oft im Schulunterricht stark unterfordert und die Schulen k\"o nnen Ihnen genau das nicht geben, was diese Sch\"u ler zur Entfaltung ihres Talents ben\"otigen: gezielte und professionale Hilfestellung bei komplexen Problemen und Fragestellungen, wie sie im Auswahlverfahren und der IPhO vorkommen.  Desweiteren erhalten die Sch\"uler die M\"oglichkeit \"uber mehr als ein Wochenende nationale Kontakte zu anderen Sch\"ulern zu kn\"upfen, die die gleichen Interessen haben.

Das diesj\"ahrige 7. Orpheus Seminar soll vom 7.10.2011 bis zum 10.10.2011 an der Technischen Universit\"at M\"unchen stattfinden. Neben den klassichen Veranstaltungen eines Vorbereitungsseminars, sind diesmal auch Vortr\"age geplant in denen TUM-Forscher Einblicke in ihre Forschung geben. Dabei wird das Themensprektrum von mathematischer Physik, \"uber Quantenfeldtheorien bis hin zu moderner angewandter Experimental- und Biophysik gehen.  Desweiteren sind Institutsrundg\"ange und ein physikfreier Sonntagabend geplant.  Die Veranstaltungen der Orpheus Betreuer werden Aufgabenseminare, Theorievorlesungen und Experimentalbetreuung umfassen.  

Die Seminare haben immer ausgesprochen positives Feedback von den Teilnehmern und Teilnehmerinnen erhalten.  Dies freut uns als Veranstalter besonders und ist auch Ansporn f\"ur uns, die zuk\"unftige T\"atigkeit auszuweiten und zu intensivieren.  Deshalb planen wir dieses Jahr, das Seminar um einen Tag zu verl\"angern.  Damit dieses Seminar auch finanziell abgesichert werden kann und die Kosten nicht nur von den Teilnehmern und den R\"ucklagen des Vereins getragen werden m\"ussen, wenden wir uns an Sie mit der Bitte um Unterst\"utzung.

Die Unterst\"utzung durch die Dr. Isolde-Dietrich-Stiftung kann sowohl ideeller als auch materieller Natur sein.  Einerseits ben\"otigen wir Hilfe f\"ur eine erfolgreiche Pressearbeit, auch im Interesse unserer F\"orderer,  andererseits ben\"otigen wir Ihre Unterst\"utzung in der finanziellen Absicherung des Seminars. Es ist uns ein wichtiges Ziel, die Teilnahme an dieser Veranstaltung nicht vom finanziellen Hintergrund abh\"angig machen, sondern aus reiner Begeisterung f\"ur die Physik m\"oglich werden zu lassen.  Wir kennen alle die Bef\"urchtungen um die wirtschaftlichen und vor allem wissenschaftlichen Verluste, die durch den Mangel an hervorragenden Physikern entstehen k\"onnen.  Aktivit\"aten wie die Physikolympiade werben f\"ur das Fach Physik und machen die eigene Leistungsf\"ahigkeit transparent. Die F\"orderung angehender Physiktalente hilft also, das langfristige physikalische Potential unseres Landes zu sichern und ist daher für die Grundlagenforschung wichtig.  In diesem Sinne kann die Dr. Isolde-Dietrich-Stiftung mit Ihrer Unterst\"utzung einen Teil zum Erfolg des Physikseminares beitragen und so den Aufwind nutzen, den die Physikolympiade oder gar eine erfolgreiche Teilnahme bei unserem Seminar bei jungen Menschen bewirkt.

Es w\"urde uns sehr freuen, wenn wir auch Ihr Interesse an unserem Projekt, vielleicht sogar in Verbindung mit einem Engagement f\"ur diese F\"orderung des physikalischen Nachwuchses, geweckt h\"atten.

Detailliertere Informationen zum Verein und dem geplanten Seminar finden Sie auf dem beiliegenden Informationsblatt.  Bei weiteren Fragen  d\"urfen Sie uns gerne kontaktieren.





\closing{Mit freundlichen Gr\"u\ss en}
% Informationen in PS verfrachten zwecks besonderer Aufmerksamkeit?
% Habe hier noch strker gekrzt, da wir ein Angebot wollen und nicht in erste Linier Sponsoring. 

% Kontaktinformationen (Tel.!) in Fuzeile \"uberpr\"ufen

\end{letter}




\end{document}
