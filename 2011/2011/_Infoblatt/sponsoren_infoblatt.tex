\documentclass[DIV13,11pt,a4paper,headinclude]{scrartcl}
\usepackage{url}
\usepackage[T1]{fontenc}
\usepackage[ansinew]{inputenc}
\usepackage[ngerman]{babel}
\usepackage{tabularx}
\usepackage{graphicx}
\usepackage{scrpage2}
\usepackage{eurosym}

\begin{document}

\pagestyle{scrheadings}


\newcommand{\fusszeile}{\par
 			 \vspace{6pt}
			 \par
 			 \tiny \normalfont
       \par \noindent\begin{tabularx}{\textwidth}{@{}p{2.5cm}Xp{2.5cm}Xp{2cm}Xp{2.2cm}@{}}
			 Orpheus e.\,V. \newline
			 \nicefrac{c}{o} Olga Goulko\newline
			 Bert-Brecht-Allee 5\newline
			 81735 M\"unchen \par
			 &&
			 \textsf{Kontakt:}\newline
			 \url{info@orpheus-verein.de} \newline
			 \url{www.orpheus-verein.de} \newline
			 Tel. 0176/40124455
			 &&
			 \textsf{Bankverbindung:} \newline
			 SSK M\"unchen \newline
			 BLZ 70150000 \newline
			 Kto.-Nr. 149435
			 
			 &&
			 \textsf{Vorstandsvorsitzender:}\newline
			 Dankrad Feist \newline
			 \textsf{Zweiter Vorsitzender:}\newline
			 Tobias Fritz
			 \end{tabularx}}
			 
\cfoot{\normalfont -- \thepage{} --}

\setheadwidth[0pt]{text}		 

\ihead{ \normalfont 

\vspace{-30pt}

{
										  \begin{tabularx}{\textwidth}{@{}X@{\hspace{3pt}}p{115pt}@{\hspace{3pt}}c@{}}
											\hrule
											%
											&
											%
											\hfill\mbox{\hfill
											%
											\raisebox{-3pt}{\begin{minipage}{81pt}
											{\tiny
											%
											\textsf{\textbf{Or}ganisationsgruppe \textbf{Ph}ysik \newline 
											f\"ur \textbf{eu}rop\"aische Sch\"uler und \newline 
											\textbf{S}tudenten (Orpheus) e.\,V.} \par} % \\
											%
											\end{minipage}
											%
											\hspace{-2pt}
											%
											\raisebox{-9pt}{\includegraphics[height=25pt]{logo.pdf}}}}
											&
											\rule{2mm}{.4pt}
											
											\end{tabularx}
										}
}

\setlength{\parindent}{0pt}
\setlength{\parskip}{0.5\baselineskip}

\section*{Informationen \"uber das Vorbereitungsseminar f\"ur Teilnehmer der Physikolympiade vom 7. bis 10. Oktober 2011 in M\"unchen}

\subsection*{Das Auswahlverfahren zur Internationalen Physikolympiade}

Die Internationale Physikolympiade (IPhO) ist ein einmal j\"ahrlich
stattfindender Sch\"ulerwettbewerb, bei dem Sch\"uler aus inzwischen \"uber 80
L\"andern ihre physikalischen F\"ahigkeiten in theoretischen und experimentellen
Klausuren unter Beweis stellen k\"onnen. Nachdem in diesem Jahr Kroatien als
Gastgeberland fungiert, wird die deutsche Mannschaft im Jahr 2011 zum
Wettbewerb nach Thailand fahren. Um die geeignetsten Sch\"uler in die deutsche
Mannschaft aufzunehmen, existiert in Deutschland ein mehrstufiges
Auswahlverfahren, an dem sich j\"ahrlich zwischen 300 und 400 Sch\"uler
beteiligen. Die Teilnahme ist komplett freiwillig und in der Regel mit
keinerlei schulischen Vorteilen verbunden.

Neben der Auswahl der deutschen Mannschaft dient der Wettbewerb auch allgemein
zur weiteren F\"orderung der naturwissenschaftlichen Interessen der Sch\"uler und
bietet ihnen die seltene M\"oglichkeit, zu gleichgesinnten Altersgenossen
Kontakte zu kn\"upfen.

Weitere Informationen zur IPhO und dem deutschen Auswahlverfahren finden sich
auf der Webseite~\cite{iphoweb}.

\subsection*{Das Seminar in M\"unchen}

Da die ersten beiden Runden des deutschen Auswahlverfahrens in selbstst\"andiger
Hausarbeit durchgef\"uhrt werden, haben die teilnehmenden Sch\"uler erst bei der
zentral veranstalteten dritten Runde die M\"oglichkeit, ihre Mitstreiter
kennenzulernen. Au\ss erdem gehen die Anforderungen an die Teilnehmer des
Wettbewerbs oft deutlich \"uber den schulischen Physikunterricht hinaus.
Aus diesen Gr\"unden bieten wir von Orpheus e.V. jeden Herbst ein zentral
organisiertes Wochenendseminar an, wo wir den Sch\"ulern durch Experimentalpraktika,
Theorievortr\"age und bungsaufgaben IPhO-relevante Themen vermitteln, die im
Schulunterricht oft nur kurz oder \"uberhaupt nicht besprochen werden. F\"ur viele
Teilnehmer ist es au\ss erdem das erste Mal, dass sie selbstst\"andig
experimentieren k\"onnen und den Umgang mit physikalischen Messger\"aten lernen.

Zum Seminar eingeladen werden diejenigen Sch\"uler, die die erste Runde des
Auswahlverfahrens erfolgreich absolviert haben. Dies ist aber kein notwendiges
Kriterium, und wir freuen uns auch \"uber andere interessierte Teilnehmer. In den
Jahren 2008-2009 hatte unser Seminar (in Berlin bzw. Kaiserslautern) jeweils etwa 70
Teilnehmer, von denen wir fast ausschlie\ss lich positive R\"uckmeldungen bekommen
haben. Bei dem Seminar in Kaiserslautern war das Interesse sogar so gro\ss , dass 
insgesamt \"uber 150 Pl\"atze h\"atten vergeben werden k\"onnen. \"Uber 80 Sch\"ulern musste leider abgesagt werden. 
Deshalb entschied sich der Verein, die 
Teilnehmerzahl beim Seminar in Rostock im letzten Jahr auf 90 zu erh\"ohen. Auch beim diesj\"ahrigen Seminar in M\"unchen soll das Ziel beibehalten werden.

Au\ss erdem war es der Wunsch vieler Teilnehmer, das Seminar zu verl\"angern. Dieser Wunsch soll dieses Jahr erf\"ullt werden: Das Seminar wird um einen Tag verl\"angert und findet zum ersten Mal vom Freitag bis Montag statt.

Das diesj\"ahrige Seminar soll vom 7. bis zum 10. Oktober 2011 an der Technische Universit\"at M\"unchen stattfinden. Es wird von dem Orpheus-Verein in enger Zusammenarbeit mit der TU M\"unchen organisiert, sowie mit Unterst\"utzung des IPhO-Landesbeauftragten Herrn StD Richard Reindl durchgef\"uhrt.

Die ehrenamtlichen Seminarbetreuer und Kursleiter von Orpheus e.V. sind selbst
alle ehemalige Teilnehmer des deutschen Auswahlverfahrens oder ehemalige
Mitglieder der deutschen Mannschaft. Dies erm\"oglicht einen guten Draht zu den
Seminarteilnehmern sowohl auf fachlicher als auch auf pers\"onlicher Ebene.

\subsection*{Programm des Seminars}

Nach derzeitigem Planungsstand sind folgende Programmpunkte vorgesehen:

\begin{itemize}
\item Freitag: 
	\begin{itemize}
	\item Anreise bis 14:30 Uhr
	\item Begr\"uung
	\item Experimentalpraktika und Theorievortr\"age bis etwa 18:30 Uhr
	\item nach dem Essen Spiele- und Kennenlernabend in der Unterkunft
	\end{itemize}
\item Samstag:
	\begin{itemize}
	\item tags\"uber weitere Experimentalpraktika und Theorievortr\"age
	\item Gespr\"ache mit Professoren verschiedener Physik-Insitute
	\item am Abend Vortr\"age zu Themen der modernen Physik
	\end{itemize}
\item Sonntag
	\begin{itemize}
	\item tags\"uber weitere Experimentalpraktika und Theorievortr\"age
	\item am Abend Vortr\"age zu Themen der modernen Physik
	\end{itemize}
\item Montag
	\begin{itemize}
	\item morgens Experimentalpraktika und Theorievortr\"age
	\item Institutsf\"uhrungen
	\item Nachmittags Abreise
	\end{itemize}	
\end{itemize}

\subsection*{Finanzielle Planung}

F\"ur einen Teilnehmer oder einen der ehrenamtlichen Kursleiter ist etwa mit folgenden Ausgaben zu rechnen:

{%\small
\begin{center}
\begin{tabular}{|l|r|}
\hline
Posten & Betrag pro Person (Sch\"atzung)\\
\hline\hline
bernachtung mit Fr\"uhst\"uck & $3\:\times 25$ \euro \\\hline
Verpflegung & 30 \euro \\\hline
Verkehrsmittel & 15 \euro \\\hline \hline
& Summe: 120 \euro \\ \hline
\end{tabular}
\end{center}}

Da das Interesse f\"ur die letzten Seminare so gro\ss war, planen wir mit 100 Teilnehmern sowie 15
ehrenamtlichen Betreuern und Organisatoren. Dies entspricht einem finanziellen Aufwand von etwa 13800\,\euro{}. Zusammen mit den Reisekosten der Betreuer (erfahrungsgem 100\,\euro{} pro Person) ergibt diese eine Gesamtsumme von etwa
\begin{displaymath}
\textrm{15300 \euro{}.}
\end{displaymath}

Diese Kosten werden -- abz\"uglich potentieller Sponsorengelder -- auf die
Teilnehmer umgelegt. Gern w\"urden wir den Sch\"ulerinnen und Sch\"ulern eine
m\"oglichst g\"unstige oder komplett kostenfreie Teilnahme erm\"oglichen, und sie
nach M\"oglichkeit sogar bei den Kosten der Anreise (je nach Wohnort zwischen 0
\euro{} und 150 \euro{}) unterst\"utzen. Dies w\"urde insbesondere allen
interessierten Sch\"ulerinnen und Sch\"ulern die M\"oglichkeit bieten, unser Seminar
in Anspruch zu nehmen.

Dieses Ziel k\"onnen wir nur mit Ihrer Unterst\"utzung erreichen.

% Bessere berschrift?
\subsection*{Sponsoring}

Wir w\"urden uns sehr freuen, falls wir Sie f\"ur ein Engagement zugunsten dieses
Projekts begeistern konnten.

F\"ur Spenden an den als gemeinn\"utzig anerkannten Verein k\"onnen wir
Zuwendungsbest\"atigungen ausstellen. Im Rahmen eines Sponsorings w\"urden wir Ihr
Engagement auf den Einladungsschreiben, dem Programm, dem Bericht auf unserer
Vereinshomepage sowie in den Pressemitteilungen \"uber das Seminar entsprechend
w\"urdigen.

Auch anderen Formen der Zusammenarbeit stehen wir aufgeschlossen gegen\"uber. Bei
Interesse an einem Engagement oder weiteren Fragen zum Seminar kontaktieren Sie
uns bitte. Wir freuen uns auf Ihre Nachricht.

\subsection*{Hintergrund von Orpheus e.V.}

Orpheus e.\,V. ist ein gemeinn\"utziger Verein, der im Jahr 2004 von ehemaligen
Teilnehmern des deutschen IPhO-Auswahlverfahrens gegr\"undet wurde. Das
derzeitige Mitgliederspektrum des Vereins reicht dabei von Sch\"ulern \"uber
Studenten hin zu Doktoranden in Physik und Mathematik. Neben unseren Seminaren
f\"uhren wir auerdem j\"ahrliche Mitgliedertreffen an wechselnden Orten durch.
Eine Teilnahme am IPhO-Auswahlverfahren ist nicht Voraussetzung f\"ur eine
Mitgliedschaft.

Kurze Berichte zu unseren Seminaren der Jahre 2005 bis 2010 gibt es auf unserer
Webseite~\cite{orpheusweb}.

% Kontaktinformationen wiederholen finde ich unn\"otig

\renewcommand*{\refname}{Referenzen} % berschrift f\"ur das Literaturverzeichnis
\begin{thebibliography}{5}

\bibitem{iphoweb} \url{www.ipho.info}.
\bibitem{orpheusweb} \url{www.orpheus-verein.de}, unter \emph{Veranstaltungen} - \emph{Seminare}.

\end{thebibliography}

\end{document}
